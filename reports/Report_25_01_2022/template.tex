\documentclass[]{article}
\usepackage[german]{babel}
\usepackage{graphicx}
\usepackage{tabularx}
\usepackage[backend=bibtex, natbib=true]{biblatex}
\usepackage{listings}
\usepackage{tikz}

\lstset{%
	basicstyle=\ttfamily\scriptsize,        % Code font, Examples: \footnotesize, \ttfamily
	keywordstyle=\color{blue!80!black},     % Keywords font ('*' = uppercase)
	commentstyle=\color{gray},              % Comments font
	numbers=left,                           % Line nums position
	numberstyle=\tiny,                      % Line-numbers fonts
	stepnumber=1,                           % Step between two line-numbers
	numbersep=5pt,                          % How far are line-numbers from code
	backgroundcolor=\color{gray!10!white},  % Choose background color
	frame=none,                             % A frame around the code
	tabsize=2,                              % Default tab size
	captionpos=b,                           % Caption-position = bottom
	breaklines=true,                        % Automatic line breaking?
	breakatwhitespace=false,                % Automatic breaks only at whitespace?
	showspaces=false,                       % Dont make spaces visible
	showstringspaces=false                  %
	showtabs=false,                         % Dont make tabls visible
	columns=flexible,                       % Column format
	morekeywords={},                        % Specific keywords
	stringstyle=\color{green!50!black},%
}%

\bibliography{bibliography}
%opening
%Here you can enter your names and titleof your report
\title{Weekly Reports}
\author{Luftqualität in Innenräumen - Gruppe 1}

\begin{document}

\maketitle

\begin{table}[h!]
	\centering
	\begin{tabular}{|c|c|c|}
		\hline
		{\textbf{Name}}				&		{\textbf{Matrikel Nr.}} & {\textbf{Arbeitsaufwand (h)}} \\
		\hline
		Friedrich Just				&		1326699 				&		18,00\\
		\hline
		Stipe Knez					&		1269206 				&	19,00	\\
		\hline
		Lucas Merkert				&		1326709					&	20,00	\\
		\hline
		Achim Glaesmann				&		1309221					&	17,50	\\
		\hline
		Max-Rene Konieczka			&		1211092					&	18,00	\\
		\hline
		Can Cihan Nazlier			&		1179244					&	25,00	\\
		\hline
	\end{tabular}
	\caption{Arbeitsaufwand dieser Woche}
	\label{tab:worakload}
\end{table}



\section{Überblick}


\subsection{Friedrich Just}







\subsection{Stipe Knez}

\subsection{Lucas Merkert}



\subsection{Achim Glaesmann}


\subsection{Max-Rene Konieczka}

\subsection{Can Cihan Nazlier}
In dieser Sonntag gab es ein Treffen mit allen Gruppenmitgliedern bei mir Zuhause. Dort wurde das zusammenspiel aus Frontend, Backend und den Sensoren mit echten Messwerten getestet. Dieses Treffen lief reibungslos und alle Komponenten haben wie erwartet miteinander kommuniziert. Daran anknüpfend wurden im Fronent info buttons zu jeweiligen Räumen wo die Sensoren liegen erstellt, die dann die echtzeit Messdaten der Sensoren im Frontend anzeigen. Das speichern der Messwerte in der Datenbank klappt auch wie erwartet.
Die Sensoren die man in die Räume hinzufügt wurden in einer seperaten Tabelle in der Datenbank gespeichert, um Positionen der jeweiligen Sensoren nicht zu verlieren.
Der letzte Schritt ist nun das Dashboard zu erstellen.

\printbibliography
%----------------------------------------------------------------------------
% Bibliography
%----------------------------------------------------------------------------	

\end{document}

\documentclass[]{article}
\usepackage[german]{babel}
\usepackage{graphicx}
\usepackage{tabularx}
\usepackage[backend=bibtex, natbib=true]{biblatex}
\usepackage{listings}
\usepackage{tikz}

\lstset{%
	basicstyle=\ttfamily\scriptsize,        % Code font, Examples: \footnotesize, \ttfamily
	keywordstyle=\color{blue!80!black},     % Keywords font ('*' = uppercase)
	commentstyle=\color{gray},              % Comments font
	numbers=left,                           % Line nums position
	numberstyle=\tiny,                      % Line-numbers fonts
	stepnumber=1,                           % Step between two line-numbers
	numbersep=5pt,                          % How far are line-numbers from code
	backgroundcolor=\color{gray!10!white},  % Choose background color
	frame=none,                             % A frame around the code
	tabsize=2,                              % Default tab size
	captionpos=b,                           % Caption-position = bottom
	breaklines=true,                        % Automatic line breaking?
	breakatwhitespace=false,                % Automatic breaks only at whitespace?
	showspaces=false,                       % Dont make spaces visible
	showstringspaces=false                  %
	showtabs=false,                         % Dont make tabls visible
	columns=flexible,                       % Column format
	morekeywords={},                        % Specific keywords
	stringstyle=\color{green!50!black},%
}%

\bibliography{bibliography}
%opening
%Here you can enter your names and titleof your report
\title{Weekly Reports}
\author{Luftqualität in Innenräumen - Gruppe 1}

\begin{document}

\maketitle

\begin{table}[h!]
	\centering
	\begin{tabular}{|c|c|c|}
		\hline
		{\textbf{Name}}				&		{\textbf{Matrikel Nr.}} & {\textbf{Arbeitsaufwand (h)}} \\
		\hline
		Friedrich Just				&		1326699 				&	1,00	\\
		\hline
		Stipe Knez					&		1269206 				&	1,00	\\
		\hline
		Lucas Merkert				&		1326709					&	1,00	\\
		\hline
		Achim Glaesmann				&		1309221					&	1,00	\\
		\hline
		Max-Rene Konieczka			&		1211092					&	1,00	\\
		\hline
		Can Cihan Nazlier			&		1179244					&	1,00	\\
		\hline
	\end{tabular}
	\caption{Arbeitsaufwand dieser Woche}
	\label{tab:worakload}
\end{table}



\section{Überblick}


\subsection{Friedrich Just}
Usually when using a \LaTeX\ document sections and paragraphs are used to categorize your content. Down below you'll find some examples how to create and use sections and parapgraphs. As you will see the enumeration of the sections and subsections will be done automatically. 

\subsection{Stipe Knez}

\subsection{Lucas Merkert}
test~\cite{influxdb_sql_db} \\

\subsection{Achim Glaesmann}

\subsection{Max-Rene Konieczka}

\subsection{Can Cihan Nazlier}
	



\subsection{Internet sources}
Sources out of the internet are often used by students, but \textbf{do not} only provide the URL of the source. You'll always need to state the author, title and your last access to the website as well! If you look in the References section down below, you'll see how to reference a web source correctly. 

\printbibliography
%----------------------------------------------------------------------------
% Bibliography
%----------------------------------------------------------------------------	

\end{document}

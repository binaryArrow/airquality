\documentclass[]{article}
\usepackage[german]{babel}
\usepackage{graphicx}
\usepackage{tabularx}
\usepackage[backend=bibtex, natbib=true]{biblatex}
\usepackage{listings}
\usepackage{tikz}

\lstset{%
	basicstyle=\ttfamily\scriptsize,        % Code font, Examples: \footnotesize, \ttfamily
	keywordstyle=\color{blue!80!black},     % Keywords font ('*' = uppercase)
	commentstyle=\color{gray},              % Comments font
	numbers=left,                           % Line nums position
	numberstyle=\tiny,                      % Line-numbers fonts
	stepnumber=1,                           % Step between two line-numbers
	numbersep=5pt,                          % How far are line-numbers from code
	backgroundcolor=\color{gray!10!white},  % Choose background color
	frame=none,                             % A frame around the code
	tabsize=2,                              % Default tab size
	captionpos=b,                           % Caption-position = bottom
	breaklines=true,                        % Automatic line breaking?
	breakatwhitespace=false,                % Automatic breaks only at whitespace?
	showspaces=false,                       % Dont make spaces visible
	showstringspaces=false                  %
	showtabs=false,                         % Dont make tabls visible
	columns=flexible,                       % Column format
	morekeywords={},                        % Specific keywords
	stringstyle=\color{green!50!black},%
}%

\bibliography{bibliography}
%opening
%Here you can enter your names and titleof your report
\title{Weekly Reports}
\author{Luftqualität in Innenräumen - Gruppe 1}

\begin{document}

\maketitle

\begin{table}[h!]
	\centering
	\begin{tabular}{|c|c|c|}
		\hline
		{\textbf{Name}}				&		{\textbf{Matrikel Nr.}} & {\textbf{Arbeitsaufwand (h)}} \\
		\hline
		Friedrich Just				&		1326699 				&	1,00	\\
		\hline
		Stipe Knez					&		1269206 				&	1,00	\\
		\hline
		Lucas Merkert				&		1326709					&	1,00	\\
		\hline
		Achim Glaesmann				&		1309221					&	1,00	\\
		\hline
		Max-Rene Konieczka			&		1211092					&	13,00	\\
		\hline
		Can Cihan Nazlier			&		1179244					&	1,00	\\
		\hline
	\end{tabular}
	\caption{Arbeitsaufwand dieser Woche}
	\label{tab:worakload}
\end{table}



\section{Überblick}


\subsection{Friedrich Just}
Usually when using a \LaTeX\ document sections and paragraphs are used to categorize your content. Down below you'll find some examples how to create and use sections and parapgraphs. As you will see the enumeration of the sections and subsections will be done automatically. 

\subsection{Stipe Knez}

\subsection{Lucas Merkert}
Verlauf der Woche: Anschließend an die Vorlesung am Dienstag bzw. Präsentationen haben wir eine interne Nachbesprechung zu unserem Pflichtenheft gehalten und das weitere Vorgehen besprochen. Am Mittwoch war ich mit Achim und Friedrich an der Hochschule und haben uns bzgl. des Sensors SHT21 informiert~\cite{datasheetsht21} und haben eine erste Recherche zum Thema Luftqualität gemacht~\cite{co2letter}. Donnerstags habe ich mich in InfluxDB und Zeitreihendatenbanken eingelesen~\cite{influxdb_sql_db}~\cite{youtube_timeseriesdatabase}. Freitags haben wir die das Pflichtenheft zusammengeschrieben mit den Recherchen der einzelnen Teammitglieder. Samstag haben wir den Projektplan überarbeitet und uns zusammen Latex angeschaut um damit die Reports zu schreiben und den ersten Report geschrieben und über GitHub geteilt.

\subsection{Achim Glaesmann}

\subsection{Max-Rene Konieczka}
Im Verlauf der Woche habe ich mich größtenteils um Installationen gekümmert um ein geeignetes Umfeld für die Programmierung sowie Dokumentation zu schaffen. Da wir für die Entwicklung der Applikation hauptsächlich JavaScript und Java verwenden werden, habe ich zwei Code-Editoren von JetBrains runtergeladen. Webstorm für die Entwicklung in Javascript und IntelliJ IDEA Ultimate für Java. Da ich mit JavaScript noch nicht allzu viel Erfahrung habe, habe ich mir vorerst Teile einer Javascript-Dokumentation durchgelesen~\cite{javascript_doc} sowie ein Tutorial angeschaut~\cite{javascript_tut}, um mich damit vertraut zu machen.

Für die Versionsverwaltung verwenden wir Git, weshalb ich noch eine Git-Umgebung einrichten musste. Um nicht mit den Commands arbeiten zu müssen habe ich mir die GitHub-Desktop App heruntergeladen. 
Anschließend habe ich mich um die Verschriftlichung der Meilensteine für das Pflichtenheft gekümmert. Zusätzlich haben wir in der Gruppe entschieden, für jegliche Dokumentationen und Reports, LaTex zu verwenden, daher habe ich TeXstudio installiert und mich damit auseinandergesetzt. 

\subsection{Can Cihan Nazlier}
Im Verlauf der Woche habe ich unserer Projekt konfiguriert und auf Github gepusht. Er besteht aus einem backend, frontend, models und einem electron Teil. Ich habe das frontend mit node so konfiguriert, dass es sich in den electron Ordner buildet und wir eine desktop application daraus erstellen können. Das backend habe ich mit express aufgesetzt und die von Herrn Merkl empfohlene Datenbank influxDB integriert. Zudem habe ich angefangen das Zeichentool zu programmieren und einen ersten Prototypen zu entwickeln. Des weiteren habe ich mich mit den Schnittstellen befasst und dem Datenaustausch zwischen den modulen und den Mockup. Ich habe erste Datentransfermodelle entwickelt und im Laufe der Woche werde ich Mockup Daten erstellen und mit diesen erste use-cases nachstellen. Ich habe mich auch noch mit Socket IO auseinandergesetzt, weil wir mit Websockets arbeiten werden, um eine reactive app zu gestalten.

\printbibliography
%----------------------------------------------------------------------------
% Bibliography
%----------------------------------------------------------------------------	

\end{document}

\documentclass[]{article}
\usepackage[german]{babel}
\usepackage{graphicx}
\usepackage{tabularx}
\usepackage[backend=bibtex, natbib=true]{biblatex}
\usepackage{listings}
\usepackage{tikz}

\lstset{%
	basicstyle=\ttfamily\scriptsize,        % Code font, Examples: \footnotesize, \ttfamily
	keywordstyle=\color{blue!80!black},     % Keywords font ('*' = uppercase)
	commentstyle=\color{gray},              % Comments font
	numbers=left,                           % Line nums position
	numberstyle=\tiny,                      % Line-numbers fonts
	stepnumber=1,                           % Step between two line-numbers
	numbersep=5pt,                          % How far are line-numbers from code
	backgroundcolor=\color{gray!10!white},  % Choose background color
	frame=none,                             % A frame around the code
	tabsize=2,                              % Default tab size
	captionpos=b,                           % Caption-position = bottom
	breaklines=true,                        % Automatic line breaking?
	breakatwhitespace=false,                % Automatic breaks only at whitespace?
	showspaces=false,                       % Dont make spaces visible
	showstringspaces=false                  %
	showtabs=false,                         % Dont make tabls visible
	columns=flexible,                       % Column format
	morekeywords={},                        % Specific keywords
	stringstyle=\color{green!50!black},%
}%

\bibliography{bibliography}
%opening
%Here you can enter your names and titleof your report
\title{\LaTeX\  Template for the Weekly Reports}
\author{David Merkl}

\begin{document}

\maketitle
\tikzstyle{block} = [rectangle, draw, fill=white!20, text width=15em, text centered, rounded corners, minimum height=4em]
\tikzstyle{line} = [draw, Latex,length=3mm, width=2mm]
\tikzstyle{cloud} = [draw, ellipse,fill=red!20, node distance=5cm, minimum height=5em]

\begin{tikzpicture}[node distance = 2cm, auto]
	% Place nodes
	\node [block, align= center] (start) {APP\_STARTUP\_STATE \\ appInitUsartManager \\initTimer};
	\node [block, below of=start] (startnetwork) {APP\_STARTJOIN\_NETWORK \\ ZDO\_StartNetworkReq};
	\node [block, below of=startnetwork] (initendpoint) {APP\_INIT\_ENDPOINT \\ initEndpoint};
	\node [block, below of=initendpoint] (inittransmit) {APP\_INIT\_TRANSMITDATA \\ initTransmitData};
	\node [block, below of=inittransmit] (appresetccsswstate) {APP\_RESET\_CCS\_SW\_STATE \\ I2C-Write: 0xFF 0x11 0xE5 0x72 0x8A \\ SW\_RESET\_REG \\ SW\_RESET\_SEQUENCE};
	\node [block, below of=appresetccsswstate] (appccshwidwriteregstate) {APP\_CCS\_HW\_ID\_WRITE\_REG\_STATE \\ I2C-Write: 0x20 \\ HW\_ID\_REG};
	\node [block, below of=appccshwidwriteregstate] (appccshwidreadstate) {APP\_CCS\_HW\_ID\_READ\_STATE \\ I2C-Read: 1 byte \\ Hardware ID has to be 0x81};
	\node [block, below of=appccshwidreadstate] (appccschangetoappstatestate) {APP\_CCS\_CHANGE\_TO\_APPSTATE\_STATE \\ I2C-Write: 0xF4 \\ BOOTLOADER\_APP\_START};
	\node [block, below of=appccschangetoappstatestate] (appccswritemeasregstate) {APP\_CCS\_WRITE\_MEAS\_REG\_STATE \\ I2C-Write: 0x01 0xTODO \\ MEAS\_MODE\_REG \\ (Measure every TODO)};
	\node [block, below of=appccswritemeasregstate] (appccswritestatusregstate) {APP\_CCS\_WRITE\_STATUS\_REG\_STATE \\ I2C-Write: 0x00\\STATUS\_REG};
	\node [block, below of=appccswritestatusregstate] (appccsreadstatusregstate) {APP\_CCS\_READ\_STATUS\_REG\_STATE \\ I2C-Read: 1 byte\\ Bit 0: error \\ Bit 3: data ready};
	\node [block, below of=appccsreadstatusregstate] (appresetscdstate) {APP\_RESET\_SCD\_STATE \\ I2C-Write: 0x3F 0x86\\ TODO ask Achim};
	\node [block, right of=appresetscdstate, node distance=8cm] (appinitsensorstate) {APP\_INIT\_SENSOR\_STATE \\ I2C-Write: 0x21 0xB1\\ TODO ask Achim \\ HAL\_StartAppTimer(30 s)};
	\node [block, above of=appinitsensorstate] (appcallforreadscdstate) {APP\_CALL\_FOR\_READ\_SCD\_STATE \\ I2C-Write:TODO\\ TODO ask Achim};
	\node [block, above of=appcallforreadscdstate] (appreadscdstate) {APP\_READ\_SCD\_STATE \\ I2C-Read: 9 bytes\\ TODO ask Achim};
	\node [block, above of=appreadscdstate] (appcallforreadtempshtstate) {APP\_CALL\_FOR\_READ\_TEMP\_SHT\_STATE\\ I2C-Write: TODO\\ TODO ask Achim};
	\node [block, above of=appcallforreadtempshtstate] (appreadtempshtstate) {APP\_READ\_TEMP\_SHT\_STATE\\ I2C-Read: 2 bytes\\ TODO ask Achim};
	\node [block, above of=appreadtempshtstate] (appcallforreadrhshtstate) {APP\_CALL\_FOR\_READ\_RH\_SHT\_STATE\\ I2C-Write: TODO\\ TODO ask Achim};
	\node [block, above of=appcallforreadrhshtstate] (appreadrhshtstate) {APP\_READ\_RH\_SHT\_STATE\\ I2C-Read: 2 bytes\\ TODO ask Achim};
	\node [block, above of=appreadrhshtstate] (appccswriteresultregstate) {APP\_CCS\_WRITE\_RESULT\_REG\_STATE\\ I2C-Write: 0x02\\ ALG\_RESULT\_DATA\_REG};
	\node [block, above of=appccswriteresultregstate] (appccsreadresultregstate) {APP\_CCS\_READ\_RESULT\_REG\_STATE\\ I2C-Read: 4 bytes (eCO2 | TVOC)};
	\node [block, above of=appccsreadresultregstate] (appausgabestate) {APP\_AUSGABE\_STATE\\ calculating outpout of SCD,SHT,CCS};
	\node [block, above of=appausgabestate] (transmit) {APP\_TRANSMIT \\ encode Message \\ send Message};
	
	
	%\node [block, right of=initendpoint, node distance=8cm] (resetCCS) {APP_RESET_CCS_SW_STATE};
	
	% Draw edges
	\draw [-latex, line width=2pt] (start) -- (startnetwork);
	
\end{tikzpicture}
%----------------------------------------------------------------------------
% Bibliography
%----------------------------------------------------------------------------	
\printbibliography
\end{document}

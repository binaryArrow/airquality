\documentclass[]{article}
\usepackage[german]{babel}
\usepackage{graphicx}
\usepackage{tabularx}
\usepackage[backend=bibtex, natbib=true]{biblatex}
\usepackage{listings}
\usepackage{tikz}

\lstset{%
	basicstyle=\ttfamily\scriptsize,        % Code font, Examples: \footnotesize, \ttfamily
	keywordstyle=\color{blue!80!black},     % Keywords font ('*' = uppercase)
	commentstyle=\color{gray},              % Comments font
	numbers=left,                           % Line nums position
	numberstyle=\tiny,                      % Line-numbers fonts
	stepnumber=1,                           % Step between two line-numbers
	numbersep=5pt,                          % How far are line-numbers from code
	backgroundcolor=\color{gray!10!white},  % Choose background color
	frame=none,                             % A frame around the code
	tabsize=2,                              % Default tab size
	captionpos=b,                           % Caption-position = bottom
	breaklines=true,                        % Automatic line breaking?
	breakatwhitespace=false,                % Automatic breaks only at whitespace?
	showspaces=false,                       % Dont make spaces visible
	showstringspaces=false                  %
	showtabs=false,                         % Dont make tabls visible
	columns=flexible,                       % Column format
	morekeywords={},                        % Specific keywords
	stringstyle=\color{green!50!black},%
}%

\bibliography{bibliography}
%opening
%Here you can enter your names and titleof your report
\title{\LaTeX\  Template for the Weekly Reports}
\author{David Merkl}

\begin{document}

\maketitle

%The abstract is used to give a short overview of your report, article, ...
\begin{abstract}
This template will give you a short overview of basic \LaTeX\ functions and serves as a template for the upcoming weekly reports as well. Further down the usage of images, tables, code listings and right citation is explained. 
\end{abstract}

\section{Introduction}
This template was written with \textbf{TeXstudio}\footnote{https://www.texstudio.org/}, which has all needed functions to write documents in \LaTeX. Furthermore you will find several websites and forums about the usage of \LaTeX. Those will help you, if you need to use anything which is not explained down below. In the references of this document you will find two URL's to websites, which can give you further information on how to use \LaTeX. 

\section{Sections and Paragraphes}
Usually when using a \LaTeX\ document sections and paragraphs are used to categorize your content. Down below you'll find some examples how to create and use sections and parapgraphs. As you will see the enumeration of the sections and subsections will be done automatically. 

\subsection{Example: Subsection}
Some text...

\subsubsection{Example: Subsubsection}
Some text...

\paragraph{Example: Paragraph}
Some text...

\subparagraph{Example: Subparagraph}
Some text...


\section{Images}
When using images in \LaTeX\ you will need the command down below, furthermore make sure to always reference the source of an used images. The \textbf{[h]} argumet will set your image on this exact place in the document, if enough space is given. If you want to reference to an image, you'll need the following command: \textbf{Figure \ref{img:fra_logo}}.
	\begin{figure}[h]
		\centering
		\includegraphics[scale=0.40]{images/logo}
		\caption{Logo of Frankfurt University of Applied Sciences}
		\label{img:fra_logo}
	\end{figure}

\section{Tables}
Down below you'll see a simple example of a table with two coloumns. If you want to reference a table, the same command applies as when referencing to an image: \textbf{Table \ref{img:fra_logo}}  
	
\begin{table}[h!]
	\centering
	\begin{tabular}{|c|c|}
		\hline
		{\textbf{Optimization Flag}}	&		{\textbf{Execution Time (ms)}} \\
		\hline
		-O0							&		809-812								\\
		\hline
		-O1							&		612-663								\\
		\hline
		-O2							&		404-406								\\
		\hline
		-O3							&		405-406								\\
		\hline
		-Os							&		339-341								\\
		\hline
	\end{tabular}
	\caption{Comparison of execution time with different optimization levels}
	\label{tab:exc_time}
\end{table}

\section{Itemization and Enumerations}
If you need to itemize or enumerate something, you can do this as following, furthermore there are little examples of how to use subitems as well. \\ % \\ is a newline command, can also be done with \newline 

\textbf{Itemization}
\begin{itemize}
	\item Item 1
	\item Item 2
	\item Item 3
		\begin{enumerate}
			\item Subitem 1
			\item Subitem 2
		\end{enumerate}
	\item Item 4
	\item ... \\
\end{itemize}

\textbf{Enumeration}
\begin{enumerate}
	\item Item 1
	\item Item 2
	\item Item 3
		\begin{enumerate}
			\item Subitem 1
			\item Subitem 2
		\end{enumerate}
	\item Item 4
	\item ...
\end{enumerate}

\section{Code Listings}
This section will give you a little example of how to use Code Listings in your \LaTeX\ document, if you want to add some Code snippets in your weekly reports.

\begin{lstlisting}[language=C,frame=single, caption = Taskhandler of Router, label = task_router] 
void APL_TaskHandler(void){
	switch(appstate) {
		case APP_INIT_STATE:
			BSP_OpenLeds();

			timer.interval = 3000;
			timer.mode = TIMER_REPEAT_MODE;
			timer.callback = timerFired;

			HAL_StartAppTimer(&timer);

			appstate = APP_RED_STATE;
			SYS_PostTask(APL_TASK_ID);
			break; 
	}
}

\end{lstlisting} 

\section{Referencing}
In a scientific report it is mandatory to use references and citate correctly. This section will show you how to reference. Normally there are two ways to reference something. For example if you'll need to reference to a certain page in a book, this could be done as following: \cite[p.233]{zigbee}, if you just want to cite the whole source it would look like this.~\cite{sharelatex} \\

It is important as well, that if you cite a specific sentence or part of something, that you use the qoute right. 

For example: "\textit{Tables are a common feature in academic writing, often used to summarize research results.}" \cite{tables}. But if you paraphraze something out of a resource, it could look something like this: Tables are often used in academic writing and can summarize research results.~\cite{tables} \\

If you want to add a new reference to your document, you need to open the \textbf{bibliography.bib} file in the \textbf{template} folder via TeXstudio and add it. In the file there are some examples for books and web sources. 

\subsection{Internet sources}
Sources out of the internet are often used by students, but \textbf{do not} only provide the URL of the source. You'll always need to state the author, title and your last access to the website as well! If you look in the References section down below, you'll see how to reference a web source correctly. 


%----------------------------------------------------------------------------
% Bibliography
%----------------------------------------------------------------------------	
\printbibliography
\end{document}

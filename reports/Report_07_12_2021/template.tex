\documentclass[]{article}
\usepackage[german]{babel}
\usepackage{graphicx}
\usepackage{tabularx}
\usepackage[backend=bibtex, natbib=true]{biblatex}
\usepackage{listings}
\usepackage{tikz}

\lstset{%
	basicstyle=\ttfamily\scriptsize,        % Code font, Examples: \footnotesize, \ttfamily
	keywordstyle=\color{blue!80!black},     % Keywords font ('*' = uppercase)
	commentstyle=\color{gray},              % Comments font
	numbers=left,                           % Line nums position
	numberstyle=\tiny,                      % Line-numbers fonts
	stepnumber=1,                           % Step between two line-numbers
	numbersep=5pt,                          % How far are line-numbers from code
	backgroundcolor=\color{gray!10!white},  % Choose background color
	frame=none,                             % A frame around the code
	tabsize=2,                              % Default tab size
	captionpos=b,                           % Caption-position = bottom
	breaklines=true,                        % Automatic line breaking?
	breakatwhitespace=false,                % Automatic breaks only at whitespace?
	showspaces=false,                       % Dont make spaces visible
	showstringspaces=false                  %
	showtabs=false,                         % Dont make tabls visible
	columns=flexible,                       % Column format
	morekeywords={},                        % Specific keywords
	stringstyle=\color{green!50!black},%
}%

\bibliography{bibliography}
%opening
%Here you can enter your names and titleof your report
\title{Weekly Reports}
\author{Luftqualität in Innenräumen - Gruppe 1}

\begin{document}

\maketitle

\begin{table}[h!]
	\centering
	\begin{tabular}{|c|c|c|}
		\hline
		{\textbf{Name}}				&		{\textbf{Matrikel Nr.}} & {\textbf{Arbeitsaufwand (h)}} \\
		\hline
		Friedrich Just				&		1326699 				&	17,00	\\
		\hline
		Stipe Knez				&		1269206 				&	17,00	\\
		\hline
		Lucas Merkert				&		1326709					&	17,00	\\
		\hline
		Achim Glaesmann				&		1309221					&	19,00	\\
		\hline
		Max-Rene Konieczka			&		1211092					&	16,00	\\
		\hline
		Can Cihan Nazlier			&		1179244					&	19,00	\\
		\hline
	\end{tabular}
	\caption{Arbeitsaufwand dieser Woche}
	\label{tab:worakload}
\end{table}



\section{Überblick}


\subsection{Friedrich Just}
In dieser Woche habe ich mich mit dem Pflichtenheft befasst, mir angesehen, welche Sensoren wir haben und was diese können. Ich habe mir die passenden Datenblätter~\cite{datasheetcss811}~\cite{datasheetscd41}~\cite{datasheetsht21} zusammengesucht. In diesen sind mir einige Begriffe unklar gewesen, wie VOC bzw. TVOC~\cite{tvoc}, eCO2~\cite{eco2} und relative Luftfeuchtigkeit~\cite{realtiveluftfeuchtigkeit}. Dazu habe ich recherchiert. Danach habe ich diese durchgearbeitet, um die Sensoren und deren Funktionsweise genauer zu verstehen. Außerdem habe ich mich mit den Funkmodulen auseinandergesetzt und wollte den Temperatursensor aus dem ZigBee-Set(LM73) und das ZigBee-Netzwerk miteinander verbinden, so dass die Sensordaten an den Koordinator weitergeleitet werden. Während dieser Zeit habe ich mich erneut mit dem I$^2$C-Bus und der Übertragung von Daten beschäftigt.
lesen 3 Bytes jeder Messwert lang 2 Byte Messauflösung und ein Byte CRC Checksum für die Fehlererkennung {Abbildung \ref{img:scd41_commandset}}
i2c Adresse 0x62 ~\cite{datasheetscd41}Seite 7
grünes Kabel SDA 
Rotes Kabel VDD
Gelbes Kabel SCL
Schwarzes Kabel GND~\cite{sht21cabelconf}

\begin{figure}[!h]
	\centering
	\includegraphics[scale=0.40]{images/scd41_umrechnung}
	\caption{Auswertung einer Messung ~\cite{datasheetscd41}}
	\label{img:scd41_umrechnung}
\end{figure}
\begin{figure}[!h]
	\centering
	\includegraphics[scale=0.40]{images/scd41_commandset}
	\caption{Befehle zum Ansteuern des SCD41 Sensors~\cite{datasheetsht21}}
	\label{img:scd41_commandset}
\end{figure}


\subsection{Stipe Knez}
In der ersten  Woche lag der Fokus auf zwei Bereichen: Dem Nachbessern des Pflichtenhefts nach einer internen Besprechnung und dem Schaffen einer geeigneten Programmierumgebung für den weiteren Verlauf des Projektes auf meinem Rechner samt der Einarbeitung in Javascript.
Neben dem ausbessern vorhandenem Inhalts haben wir uns Gedanken über unsere Vorgehensweise gemacht woraufhin ich nach Vorgehensweisen recherchiert habe~\cite{scrum}, von denen wir uns für unser eigenes Projekt inspirieren lassen können. Anschließend habe ich unsere Vorgehensweise nach kurzer Absprache und Beratung mit der Gruppe ausformuliert und für das Pflichtenheft vorbereitet.

Des Weiteren spielte wie zuvor erwähnt auch das Schaffen einer geeigneten Programmierumgebung samt der Einarbeitung in Javascript diese Woche für mich eine große Rolle. Zuerst habe ich die beiden IDEs Webstorm und IntelliJ IDEA Ultimate eingerichtet. Dabei soll Webstorm der Entwicklung in Javascript und IntelliJ der Entwicklung in Java dienen. Weil ich in Javascript noch nicht allzu viel Programmiererfahrung gesammelt habe, habe ich mich in die Sprache eingearbeitet. Hilfreich waren dabei eine Javascript Dokumentation~\cite{JS_docu} sowie Lerninhalte in Videoform~\cite{javascript_tut}. Außerdem habe ich noch die GitHub Desktop Anwendung auf meinem Rechner eingerichtet und mich mit LaTex auseinandergesetzt und die dazugehörigen Anwendungen installiert.

\subsection{Lucas Merkert}
Einarbeitung SHT21: Der SHT21 Sensor wird über den I$^2$C-Bus angesprochen {Abbildung \ref{img:sht21_commandset}} um die Temperatur und relative Luftfeuchtigkeit zu messen. Der Sensor gibt die Temperatur in einer 14 Bit Auflösung und die Relative Luftfeuchtigkeit in einer 12 Bit Auflösung zurück. Die Werte können dann mit den Formel aus {Abbildung \ref{img:sht21_tempformula}} und {Abbildung \ref{img:sht21_rhformula}} berechnet werden. Dabei ist das Problem aufgetreten wie genau man den Sensor über HAL\_WriteI2cPacket() anspricht. Mit der Lösung aus dem Forumpost sollte sich dies geklärt haben.
\begin{figure}[h]
	\centering
	\includegraphics[scale=0.60]{images/sht21_commandset}
	\caption{Befehle zum Ansteuern des SGT21 Sensors, T für Temperatur, RH für relative Luftfeuchtigkeit~\cite{datasheetsht21}}
	\label{img:sht21_commandset}
\end{figure}
\begin{figure}[h]
	\centering
	\includegraphics[scale=0.60]{images/sht21_tempformula}
	\caption{Formel zur Berechnung der Temperatur~\cite{datasheetsht21}}
	\label{img:sht21_tempformula}
\end{figure}
\begin{figure}[h]
	\centering
	\includegraphics[scale=0.60]{images/sht21_rhformula}
	\caption{Formel zur Berechnung der relativen Luftfeuchtigkeit~\cite{datasheetsht21}}
	\label{img:sht21_rhformula}
\end{figure}

\textbf{Luftqualitäts Faktoren}~\cite{faktoren_luftquali}
\begin{itemize}
	\item Luftfeuchte
	\begin{enumerate}
		\item $>23$\% =$>$ Feuchtigkeitsverlust beim Atmen nur noch bedingt kompensierbar, Bauschäden, höhere Chance auf Elektroschocks
		\item $>40$\% =$>$ austrocknen der Haut, Schleimhaut und Augen
		\item $<60$\% =$>$ Schimmelbildung =$>$ Asthma, Allergien
		\item $<80$\% =$>$ optimale Feuchtigkeit für Milben, Parasiten, Pilzen
	\end{enumerate}
	\item VOC
	\begin{enumerate}
		\item Emissionsquellen: Bauprodukte, Möbel, Lack, Lösungsmittel verdunsten, Tabakrauch, Menschen, Tiere, Mikroorganismen höheres Risiko bei Neubau/Renovierung
		\item Schäden. Geruchsbelästigung, Atemwegs-/Augenreiz, Schädigung des Nervensystems, Allergien, Krebs, Erbgutschädigung, Fortpflanzungsschädigen
		\item RW 1: lebenslange Aussetzung führt zu keine Auswirkung (>0,3mg/m$^3$)
		\item RW 2: Schäden bei anfälligen Leuten sind zu erwarten
		\item RW gilt für alle Räume in denen keine Gefahrenstoffe verwendet werden
		\item VOC misst die gesamt Anzahl der Partikel in der Luft, einzelne Partikel könne daraus nicht abgelesen werden
	\end{enumerate}
	\item Feinstaub
	\begin{enumerate}
		\item Arten von Feinstaub: PM10, PM2.5, PM0.1 (Mikrometer)
		\item Je kleiner die Partikel desto weiter können diese in den Körper, vor allem die Lunge eindringen und diese beschädigen
		\item Emissionsquellen: Tabakrauch, Kerzenruß, Kochen, Computer, Drucker, Heizen ohne lüften
		\item Maßnahmen: saugen, wischen(nass!), lüften, nicht mehr als 22C heizen, Dunstabzugshaube in der Küche, Luftreiniger
	\end{enumerate}
\end{itemize}

\subsection{Achim Glaesmann}
In der vergangenen Woche wurden von mir folgende Aufgaben bearbeitet. Zunächst wurde das Pflichtenheft ausgebessert, wobei von mir die Projektbeschreibung angefertigt wurde.
Außerdem war es nötig die für die Entwicklung vorraussichtlich benötigte Software zu installieren. 
Dazu zählte die Installation von Github, so wie eine entsprechende Einarbeitung. Die Installation von Intellij IDEA als IDE für die Java Entwicklung sowie eine entsprechende Einarbeitung.
Die Installation von WebStorm als IDE für die Javascript Entwicklung so wie eine entsprechende Einarbeitung. Die Installation von MikTex als Compiler für Tex files so wie eine entsprechende 
Einarbeitung in die Syntax von LaTex. Weiterhin wurde sich in einer kleinen Gruppe Mittwochs getroffen um die Sensoren in Verbindung mit den Mikrokontrollern zu testen. Bis 
jetzt war es uns nicht möglich die Daten auszulesen, es ist beabsichtigt das Problem in der kommenden Woche zu lösen. Eine ausgiebige Recherche der Datenblätter sollte hierbei
helfen.~\cite{datasheetsht21}q Es wurden weiterhin Recherchen betrieben zur Risikoabschätzung der Aerosolbelastung basierend auf dem CO2 gehalt der Umgebungsluft. Hierbei wurden mehrere Paper gelesen wobei 
eines bis jetzt die vielversprechensten Informationen lieferte. ~\cite{co2letter} Die Recherche wird in den kommenden Wochen fortgeführt. Weiterhin wurde die Präsentation zu Analog Digital Wandlern 
angefertigt. Das Übungsblatt 2 wurde korrigiert. Es wurden insgesamt 4 Meetings mit der Gruppe gehalten.

\subsection{Max-Rene Konieczka}
Aufbauend zur letzten Woche, hat man sich mit der korrekten Einrichtung des Projektes beschäftigt, welches von Can Cihan Nazlier letzte Woche konfiguriert wurde. Darüber hinaus wurde influxDB installiert, was sich gut dafür eignet Zeitreihen-Daten zu verwalten. Da die Applikation eine Reactive Web App sein wird, wurden Recherchen zum Thema Websockets und SocketIO gemacht. SocketIO ist eine JavaScript-Bibliothek, welche für Echtzeit-Webanwendungen verwendet wird. Diese ermöglicht bidirektionale Echtzeit-Kommunikation zwischen dem Browser und einem Server. Dadurch werden Benutzereingaben schneller behandelt und die App läuft flüssiger. Websocket wiederum ist ein Netzwerkprotokoll, was auf TCP basiert. SocketIO macht sich das Websocketprotokoll zunutze.

\subsection{Can Cihan Nazlier}
Diese Woche wurde der Prototyp des Zeichentools fertiggestellt, nachdem mit Herrn Krauße über Einzelheiten am Dienstag gesprochen wurde. Man kann nun Räume erstellen, diese abspeichern(noch nicht persistent), und sie ändern auf dem zeichenboard die farbe sobald sie abgespeichert wurden. Im Verlauf nächster Woche wird daran gearbeitet die Räume persisten abzuspeichern, sei es in eine .json Datei oder in einer Datenbank. Nächste Woche wird dazu noch am Menü gearbeitet, welches gespeicherte Räume anzeigen soll und der Nutzer diese bearbeiten können soll. 
Eine geeignete Ticketumgebung um die Übersicht für die Entwickler zu erhalten wurde konfiguriert. Benutzt wird die Applikation 'Trello'. Zwei Boards wurden eingerichtet, einmal für den backend Teil und einmal für den Frontend Teil.

\printbibliography
%----------------------------------------------------------------------------
% Bibliography
%----------------------------------------------------------------------------	
\end{document}

\documentclass[]{article}
\usepackage[german]{babel}
\usepackage{graphicx}
\usepackage{tabularx}
\usepackage[backend=bibtex, natbib=true]{biblatex}
\usepackage{listings}
\usepackage{tikz}

\lstset{%
	basicstyle=\ttfamily\scriptsize,        % Code font, Examples: \footnotesize, \ttfamily
	keywordstyle=\color{blue!80!black},     % Keywords font ('*' = uppercase)
	commentstyle=\color{gray},              % Comments font
	numbers=left,                           % Line nums position
	numberstyle=\tiny,                      % Line-numbers fonts
	stepnumber=1,                           % Step between two line-numbers
	numbersep=5pt,                          % How far are line-numbers from code
	backgroundcolor=\color{gray!10!white},  % Choose background color
	frame=none,                             % A frame around the code
	tabsize=2,                              % Default tab size
	captionpos=b,                           % Caption-position = bottom
	breaklines=true,                        % Automatic line breaking?
	breakatwhitespace=false,                % Automatic breaks only at whitespace?
	showspaces=false,                       % Dont make spaces visible
	showstringspaces=false                  %
	showtabs=false,                         % Dont make tabls visible
	columns=flexible,                       % Column format
	morekeywords={},                        % Specific keywords
	stringstyle=\color{green!50!black},%
}%

\bibliography{bibliography}
%opening
%Here you can enter your names and titleof your report
\title{Weekly Reports}
\author{Luftqualität in Innenräumen - Gruppe 1}

\begin{document}

\maketitle

\begin{table}[h!]
	\centering
	\begin{tabular}{|c|c|c|}
		\hline
		{\textbf{Name}}				&		{\textbf{Matrikel Nr.}} & {\textbf{Arbeitsaufwand (h)}} \\
		\hline
		Friedrich Just				&		1326699 				&		\\
		\hline
		Stipe Knez				&		1269206 				&		\\
		\hline
		Lucas Merkert				&		1326709					&	20,00	\\
		\hline
		Achim Glaesmann				&		1309221					&		\\
		\hline
		Max-Rene Konieczka			&		1211092					&		\\
		\hline
		Can Cihan Nazlier			&		1179244					&		\\
		\hline
	\end{tabular}
	\caption{Arbeitsaufwand dieser Woche}
	\label{tab:worakload}
\end{table}



\section{Überblick}


\subsection{Friedrich Just}

\subsection{Stipe Knez}

\subsection{Lucas Merkert}
Über die Weihnachtsferien: Einarbeitung in die Funktionalität des Sensors CCS811:
\begin{itemize}
	\item Schreiben auf den Sensor über die Adresse 0x5A
	\begin{enumerate}
		\item Schreiben des Befehls 0x40 in das Register 0x01
		\item Bisher return 0 von HALWriteI2CPacket()
	\end{enumerate}
	\item Lesen des Sensor über die Adresse 0X5B
		\begin{enumerate}
			\item Lesen von 4 Bytes zum Lesen der CO2 und TVOC werte im Register 0x02
			\item Bisher return 0 von HALReadI2CPacket()
		\end{enumerate}
	\end{itemize}

Der Wake-Pin ist zurzeit an GND angeschlossen. Allerdings scheint noch etwas nicht zu funktionieren, diesen soll in der nächsten Woche geklärt werden. 
\begin{figure}[h]
	\centering
	\includegraphics[scale=0.30]{images/i2c_ccs811}
	\caption{Funktionsweise der I2C-Packet Übertragung\cite{datasheetcss811}}
	\label{img:I2C_ccs811}
\end{figure}


\subsection{Achim Glaesmann}

\subsection{Max-Rene Konieczka}

\subsection{Can Cihan Nazlier}

\printbibliography
%----------------------------------------------------------------------------
% Bibliography
%----------------------------------------------------------------------------	

\end{document}
